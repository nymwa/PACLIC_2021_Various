\documentclass[11pt]{article}
\usepackage{paclic35}
\usepackage{times}
\usepackage{latexsym}
\usepackage{amsmath}
\usepackage{multirow}
\usepackage{url}
\DeclareMathOperator*{\argmax}{arg\,max}
\setlength\titlebox{6.5cm}    % Expanding the titlebox

\usepackage{natbib}
\usepackage{amsmath,amssymb,amsthm}
\usepackage{bm}
\usepackage{graphics}
\usepackage{graphicx}
\usepackage{color}
\usepackage{xcolor}
\usepackage[cmtip,all]{xy}
\usepackage{hyperref}
\definecolor{darkblue}{rgb}{0, 0, 0.5}

\usepackage{tikz}
\usetikzlibrary{automata}
\usetikzlibrary{arrows}
\usetikzlibrary{arrows.meta}
\usetikzlibrary{positioning}
\usetikzlibrary{intersections, calc}
\usetikzlibrary{decorations}
\usetikzlibrary{decorations.markings}
\usetikzlibrary{decorations.pathreplacing,angles,quotes}
\usetikzlibrary{fit}
\usetikzlibrary{math}
\usetikzlibrary{shapes}
\usepackage{pgfplots}
\pgfplotsset{compat=1.17}
\usepackage{bchart}

\begin{document}

\appendix

\section{Integration of Our Method and RTT}

\begin{figure}[t]
	\centering
	\begin{tikzpicture}[
			>=latex,
			font=\sffamily,
			scale=0.7,
			transform shape]
		\node at (0, 0) {\textcolor{blue}{$s$}};
		\node[rectangle, text width=60, text height=10, rounded corners, draw=green, fill=green!20, outer sep=1mm] (de) at (3, 1) {};
		\node[rectangle, text width=60, text height=10, rounded corners, draw=green, fill=green!20, outer sep=1mm] (fi) at (3, 2) {};
		\node[rectangle, text width=60, text height=10, rounded corners, draw=green, fill=green!20, outer sep=1mm] (fr) at (3, 3) {};
		\node[rectangle, text width=60, text height=10, rounded corners, draw=green, fill=green!20, outer sep=1mm] (lv) at (3, 4) {};
		\node[rectangle, text width=50, text height=15, rounded corners, draw=green, fill=green!20, outer sep=1mm] (rule) at (6.7, 0) {};
		\node at (6.7, 0.2) {rule-based};
		\node at (6.7, -0.2) {AEG};
		\node at (3, 1) {En$\to$De$\to$En};
		\node at (3, 2) {En$\to$Fi$\to$En};
		\node at (3, 3) {En$\to$Fr$\to$En};
		\node at (3, 4) {En$\to$Lv$\to$En};
		\node at (2, 0.2) {$1-p$};
		\node at (1.2, 0.95) {$\frac{p}{4}$};
		\node at (1.2, 1.60) {$\frac{p}{4}$};
		\node at (1.2, 2.25) {$\frac{p}{4}$};
		\node at (1.2, 2.95) {$\frac{p}{4}$};
		\draw[->] (0.2, 0) -- (5.65, 0);
		\draw[->] (0.2, 0) -- (de.west);
		\draw[->] (0.2, 0) -- (fi.west);
		\draw[->] (0.2, 0) -- (fr.west);
		\draw[->] (0.2, 0) -- (lv.west);
		\draw[->] (de.east) -- (5.65, 0);
		\draw[->] (fi.east) -- (5.65, 0);
		\draw[->] (fr.east) -- (5.65, 0);
		\draw[->] (lv.east) -- (5.65, 0);
		\draw[->] (rule.east) -- (8.3, 0);
		\node at (8.5, 0) {\textcolor{red}{$s'$}};
		\node at (0, -0.5) {\textcolor{blue}{corrected}};
		\node at (8.5, -0.5) {\textcolor{red}{erronous}};
	\end{tikzpicture}
	\caption{Integration of Our Method and RTT}
	\label{fig:comb}
\end{figure}

\begin{table}[t]
	\scriptsize
	\centering
	\tabcolsep 4pt
	\begin{tabular}{lccc}
		\hline
		$p$
		& \hspace{-1em}{$\def\arraystretch{0.5}\begin{array}{c}\vspace{-0.5em}\\\text{BEA-19}\\\text{test}\\\end{array}$}\hspace{-1em}
		& \hspace{-1em}{$\def\arraystretch{0.5}\begin{array}{c}\vspace{-0.5em}\\\text{CoNLL}\\\text{14}\\\end{array}$}\hspace{-1em}
		& \hspace{-1em}{$\def\arraystretch{0.5}\begin{array}{c}\vspace{-0.5em}\\\text{JFLEG}\\\text{test}\\\end{array}$}\hspace{-1em}
		\\ \hline
		0.0
		& 67.32 / 69.42
		& 60.60 / 62.25
		& 60.12 / 60.69 \\
		0.1
		& 67.52 / 69.28
		& 60.73 / 62.78
		& 60.21 / 60.84 \\
		0.2
		& 67.35 / \textbf{69.83}
		& 60.58 / 62.69
		& 60.19 / 60.69 \\
		0.3
		& 66.60 / 68.58
		& 60.37 / 61.78
		& \textbf{60.38} / \textbf{60.91} \\
		0.4
		& \textbf{68.00} / 69.76
		& 60.48 / 62.13
		& 59.90 / 60.46 \\
		0.5
		& 67.92 / 69.71
		& 60.55 / 62.02
		& 60.15 / 60.74 \\
		0.6
		& 67.28 / 69.02
		& 60.75 / 61.65
		& 60.20 / 60.60 \\
		0.7
		& 67.80 / 69.38
		& \textbf{61.29} / \textbf{63.47}
		& 60.26 / 60.75 \\
		0.8
		& 67.94 / 69.53
		& 60.22 / 61.81
		& 60.32 / 60.74 \\
		0.9
		& 67.09 / 69.25
		& 60.49 / 62.95
		& 60.04 / 60.60 \\
		1.0
		& 67.65 / 69.08
		& 60.63 / 61.44
		& 60.21 / 60.57 \\
		\hline
	\end{tabular}
	\caption{Integration of our method and RTT.}
	\label{tab:integrate}
\end{table}

We further investigate if combination of our method and RTT is beneficial.
We integrate them by applying RTT at the probability of $p$, in which RTT'ed sentence is selected with an equal probability, then conducting our method.
Figure~\ref{fig:comb} shows the integrated procedure.

Table~\ref{tab:integrate} shows the results.
The optimal $p$ varies for each evaluation dataset, and it is difficult to say if the integration is beneficial.
There might be a good method to combine our method and MT-based methods.
We plan to research this in future work.

\section{Effectiveness of BPE-dropout}

\begin{table}[t]
	\scriptsize
	\centering
	\tabcolsep 4pt
	\begin{tabular}{lccc}
		\hline
		$p$
		& \hspace{-1em}{$\def\arraystretch{0.5}\begin{array}{c}\vspace{-0.5em}\\\text{BEA-19}\\\text{test}\\\end{array}$}\hspace{-1em}
		& \hspace{-1em}{$\def\arraystretch{0.5}\begin{array}{c}\vspace{-0.5em}\\\text{CoNLL}\\\text{14}\\\end{array}$}\hspace{-1em}
		& \hspace{-1em}{$\def\arraystretch{0.5}\begin{array}{c}\vspace{-0.5em}\\\text{JFLEG}\\\text{test}\\\end{array}$}\hspace{-1em}
		\\ \hline
		0.0
		& 55.21 / 61.21
		& 49.68 / 53.45
		& 53.73 / 54.26 \\
		0.05
		& 58.87 / \textbf{63.56}
		& 53.26 / 56.51
		& 57.19 / 57.78 \\
		0.1
		& \textbf{59.09} / \textbf{63.56}
		& \textbf{54.03} / \textbf{56.79}
		& \textbf{57.33} / \textbf{58.15} \\
		0.15
		& 58.62 / 63.27
		& 53.31 / 56.36
		& 57.25 / 57.73 \\
		0.20
		& 57.59 / 62.60
		& 53.54 / 56.42
		& 57.24 / 57.82 \\
		\hline
	\end{tabular}
	\caption{The effect of BPE-dropout for target-only settings.}
	\label{tab:drop1}
\end{table}

\begin{table}[t]
	\scriptsize
	\centering
	\tabcolsep 4pt
	\begin{tabular}{lccc}
		\hline
		$p$
		& \hspace{-1em}{$\def\arraystretch{0.5}\begin{array}{c}\vspace{-0.5em}\\\text{BEA-19}\\\text{test}\\\end{array}$}\hspace{-1em}
		& \hspace{-1em}{$\def\arraystretch{0.5}\begin{array}{c}\vspace{-0.5em}\\\text{CoNLL}\\\text{14}\\\end{array}$}\hspace{-1em}
		& \hspace{-1em}{$\def\arraystretch{0.5}\begin{array}{c}\vspace{-0.5em}\\\text{JFLEG}\\\text{test}\\\end{array}$}\hspace{-1em}
		\\ \hline
		0.0
		& 66.51 / 69.34
		& 60.36 / \textbf{63.14}
		& 59.55 / 60.04 \\
		0.05
		& 67.19 / \textbf{70.32}
		& 59.68 / 62.73
		& 60.41 / \textbf{61.03} \\
		0.1
		& \textbf{67.32} / 69.42
		& \textbf{60.60} / 62.25
		& 60.12 / 60.69 \\
		0.15
		& 66.90 / 69.26
		& 60.53 / 61.97
		& 60.31 / 60.71 \\
		0.20
		& 66.96 / 69.34
		& 60.32 / 62.28
		& \textbf{60.45} / 60.95 \\
		\hline
	\end{tabular}
	\caption{The effect of BPE-dropout for artificial+target settings.}
	\label{tab:drop2}
\end{table}

All the experiments in our paper use BPE-dropout of $p=0.1$, which is recommended in the original paper.
We investigate the effectiveness of subword regularization with changing $p$.

Table~\ref{tab:drop1} shows the result for baseline models.
The optimal $p$ is $0.1$.
BPE-dropout improves performance as subword reguralization.

Table~\ref{tab:drop2} shows the result for artificial+target settings using 16M augmented data.
BPE-dropout applies to both pre-training and fine-tuning data at the same dropout probability.
The optimal $p$ varies for each evaluation dataset.
Furthermore, we can observe that BPE-dropout improves even pre-trained and fine-tuned models in BEA-19 test and JFLEG test dataset.

\end{document}
